\newpage
\begin{thebibliography}{99}
\bibitem{RFC2081} J. Klensin, Simple Mail Transfer Protocol, 2001. [HTML] http://www.faqs.org/rfcs/rfc2821.html 
\bibitem{RFC2822} P. Resnick, Internet Message Format, 2001. [HTML] http://www.faqs.org/rfcs/rfc2822.html
\bibitem{MCMILLAN} Robert McMillan, What will stop spam?, SunWorld 1997. [HTML] http://sunsite.uakom.sk/sunworldonline/swol-12-1997/swol-12-vixie.html
\bibitem{SPF} Sender Policy Framework. Introduction.  [HTML] http://www.openspf.org/Introduction
\bibitem{GREYLISTING} Evan Harris, The Next Step in the Spam Control War: Greylisting, 2004 [HTML] http://projects.puremagic.com/greylisting/whitepaper.html
\bibitem{RAZOR} Оффициальная документация проекта RAZOR. [HTML] http://razor.sourceforge.net/
\bibitem{PLANFORSPAM} P. Graham, A Plan For Spam, 2002 [HTML] http://www.paulgraham.com/spam.html
\bibitem{ROBINSON} Gray Robinson, A Statistical Approach to the Spam Problem, Linux Journal, 2003 [HTML] http://www.linuxjournal.com/article/6467 
\bibitem{YURYSVM} Юрий Лифшиц, "Лекции для интернета: метод опорных векторов". [PDF] http://yury.name/internet/07ianote.pdf
\bibitem{GPUSVM}A. Athanasopoulos, A. Dimou, V. Mezaris, I. Kompatsiaris, "GPU Acceleration for Support Vector Machines", Proc. 12th International Workshop on Image Analysis for Multimedia Interactive Services (WIAMIS 2011), Delft, The Netherlands, April 2011 [PDF] http://mklab.iti.gr/files/wiamis11.pdf
\bibitem{SAPC} Набор сообщений для тестирования SpamAssassin public corpus [HTML] http://spamassassin.apache.org/publiccorpus/
\bibitem{CEAS} Набор сообщений для тестирования CEAS 2008 Live Spam Challenge Laboratory Corpus [HTML] http://plg.uwaterloo.ca/~gvcormac/ceascorpus/
\bibitem{PETROV10} А. Петров, Разработка и реализация модифицированной подсистемы байес-анализа в системе SpamAssassin, Курсовая работа, Московский Государственный университет, 2010
\bibitem{ROZ} А. Розинкин, Система защиты от массовых несанкционированных рассылок электронной почты на основе методов DATA MINING, Диссертация на соискание ученой степени кандидата физико-математических наук, москва 2010 
\end{thebibliography}
