\ESKDappendix{}{Свойства точек, измеряемые лазерными локаторами}
\label{point properties}

Все модели лазерных локаторов формируют данные с географическими координатами обнаруженных точек. В силу существования достаточно надежных алгоритмов сегментации точек на принадлежащие поверхности земли и иным предметам, по данным лазерной локации возможно построение цифровой модели высот (DEM). Это позволяет для каждой точки вычислить её высоту над уровнем земли. Исследования
\cite{Amin}
показали, что это один из наиболее важных признаков.
Вторым всегда присутствующим признаком является яркость отраженного луча. Как правило, в лазерных локаторах используются инфракрасные лазеры, а значит этот признак является яркостью точки в инфракрасном диапазоне волн.

Многие локаторы способны фиксировать множественные отклики, возникающие когда луч, попавший на границу объекта, частично отражается от него, а частично проходит дальше и отражается от следующего объекта. Наличие точек с множественным откликом характерно для растительности. Но доля точек с множественным откликом относительно невелика, поэтому данный признак как правило заменяется разбросом высот (см. приложение \ref{local properties})

Развитие емких скоростных устройств записи позволило записывать информацию о форме\footnote{То есть зависимость мощности отклика от времени} отклика, а не сам факт его наличия. Такие локаторы называют полноволновыми. Как правило, для отклика вычисляют его длину и амплитуду. Амплитуда является аналогом яркости, а длина для объемных целей говорит о их протяженности, а о непрозрачных об угле, под которым луч упал на объект.

Дополнительно к лазерной локации может производиться аэрофотосъемка местности. Она позволяет получить информацию о цвете точек. В том числе может применяться съемка в диапазонах, не видимых человеческим глазом.
Полноволновые лазерные локаторы вместо 

\ESKDappendix{}{Коллективные признаки}
\label{local properties}

Наиболее часто коллективные признаки применяются для описания геометрической формы выбранного множества точек. Наиболее популярным является перепад высот между самой высокой и низкой точками. Это связано с тем, что когда лазерный луч попадает на дерево, он может отражаться от верхних веток, а может проходить достаточно глубоко в крону. Соответственно значение этого признака для деревьев оказывается значительно больше, чем для иных объектов.

В более поздних работах\cite{full waveform} вместо этого признака предлагается использовать расстояние до плоскости, аппроксимирующей выбранное множество точек. Это позволяет лучше отличать наклонные крыши домов от зарослей кустарника, в которых точки располагаются хаотично. Кроме того, угол отклонения нормали этой плоскости от вертикали также применяется для классификации, так как скаты которых имеют характерные углы наклона.  
Как правило применяются аппроксимация плоскостью и квадратичной поверхностью. В последнем случае также учитывается особенности поверхности. \cite{Brenner}

Более сложным признаком, описывающим локальную форму объекта являются гистограммы локальной формы\cite{MMM}.
Производится вычисление главной плоскости в окрестности точки и переход в систему локальную систему координат для обеспечения независимости от поворота. Окрестность точки разделяется на блоки и вычисляется количество точек, попавших в каждый блок. В силу большой размерности получаемого признака, его использование возможно только совместно с методами понижения размерности.

Другой группой коллективных признаков являются конечные разности, минимумы, максимумы, средние значения, отклонения от средних значений признаков, измеренных в точках.

\ESKDappendix{}{Тестовые наборы}
\label{test data}

В качестве тестовых наборов использовались окрестности аэропорта Фичбург. На данных присутствуют дороги, растительность (кусты и деревья), здания (жилые коттеджи и павильоны технического назначения). Это полный набор классов, обычно используемых при классификации данных лазерной локации. Из данных было выбрано три непересекающихся набора: два набора с фрагментами коттеджной застройки и один фрагмент с павильонами и автостоянкой.
\begin{figure}[h]
\centering
\includegraphics[width=15cm]{img/fitchburg}
\caption{Аэропорт Фичбург. Ландшафт типичен для средней полосы. Присутствуют все типовые классы.}
\end{figure}
Набор \ref{sets}a использовался для обучения алгоритмов. Для определения их обобщающей способности использовались остальные два набора. \ref{sets}b примыкает к обучающему набору и содержит весьма похожие объекты. \ref{sets}c, наоборот, содержит ландшафт значительно отличающейся от обучающей выборки, поэтому на нем наблюдается падение точности классификации, по которому и оценивается обобщающая способность.
\begin{figure}[h]
\centering
\subfloat[]{\includegraphics[width=5cm]{img/learnset}\label{train set}}\qquad
\subfloat[]{\includegraphics[width=5cm]{img/testset}\label{test set 1}}\qquad
\subfloat[]{\includegraphics[width=5cm]{img/test2set}\label{test set 2}}
\caption{Тестовые наборы. Желтым отмечена земля, фиолетовым дорожное покрытие, красным дома, зеленым деревья, сиреневым кусты.}
\label{sets}
\end{figure}
\begin{figure}[h]
\centering
\includegraphics[width=15cm]{img/testperspective}
\caption{Второй тестовый набор, перспективная проекция.}
\end{figure}