\newpage
\section{Описание практической части}

% Если в рамках работы проводится реализация некоторого программного средства,
% то в разделе «Описание практической части» обязательно должна быть описана
% его программная реализация, в частности:
% приведены обоснования выбранного инструментария;
% приведена с иллюстрацией общая архитектура разработанного средства;
% приведена с иллюстрацией схема работы средства;
% если осуществляется доработка существующего средства, то должны быть описаны 
% новые возможности/улучшения, реализованные в данной работе.
% обязательно должны быть приведены характеристики функционирования (например, 
% сложность, производительность, время реакции и т.д.)

Для проверки работоспособности предложенного в разделе \ref{research} подхода была разработана экспериментальная система многоэтапной классификации данных лазерной локации. Система состоит из двух программ с графическим интерфейсом: визуального классификатора облаков точек, предназначенного для создания обучающих наборов, анализа и ручной корректировки результатов автоматической классификации, и программы многоэтапной классификации, обеспечивающей редактирование и исполнение графа классификации.

В системе имеется интерфейс для подключения внешних модулей классификации, использующий для описания прецедентов формат ARFF. В частности, этот интерфейс совместим с библиотекой алгоритмов weka, включающей такие алгоритмы, как деревья решений, метод опорных векторов, Байесовские сети, нейронные сети, максимизации ожидания. 

\subsection{Инструментарий разработки}

Система написана на языке C++ с использованием фреймворка QT4, библиотеки чтения LAS файлов liblas и библиотеки алгоритмов классификации weka. Использование QT позволяет создавать кроссплатформенные приложения с богатым графическим интерфейсом. Кроме того QT предоставляет средства для порождения нитей, запуска дочерних приложений и создания каналов. Так как разработка системы велась под ОС Linux, использование соответствующих средств ОС напрямую означало бы отсутствие возможности запуска системы по ОС, не поддерживающими POSIX. Библиотека liblas обеспечивает чтение файлов формата LAS. Это единственная открытая кроссплатформенная библиотека для работы с этим форматом. Отличается стабильностью и простотой работы. Основном языком программирования и для QT и для liblas является C++, поэтому он был выбран в качестве языка для реализации описываемой системы. 

В виду отсутствия общепринятых тестовых наборов и общепринятой методики тестирования (см. раздел \ref{datasets abscense}) для сравнения предложенного подхода к классификации со способами, предложенными другими авторами, требовалась поддержка максимального числа различных алгоритмов классификации. Поэтому в системе многоэтапной классификации был предусмотрен интерфейс для подключения внешних модулей классификации, совместимый с библиотекой алгоритмов weka.

\subsection{Архитектура}
Система состоит из модулей загрузки и сохранения данных, подсистемы многоэтапной классификации и модуля визуального контроля и коррекции. При загрузке данных производится переупорядочивание точек с целью обеспечения быстрого доступа к ним. Также производится построение цифровой модели высот (DEM). Эти данные заносятся в хранилище, поддерживающее версионирование: результат каждого этапа классификации хранится отдельно, что позволяет одновременно обрабатывать несколько этапов. Данные из этого хранилища используют подсистема многоэтапной классификации и модуль визуальной коррекции и контроля. Особенностью хранилища является то, что в файловой системе оно представлено в виде двух файлов: LAS, с точками, упорядоченными специальным образом и файла с индексом. Таким образом, любая программа, способная работать с форматом LAS может читать данные из этого хранилища.

Подсистема многоэтапной классификации состоит из управляющих модулей (контроллера и обучения) и модулей, обеспечивающих выполнение отдельных этапов классификации. Контроллер отвечает за передачу данных между этапами и вызов действий по выполнению этапа, когда поступает достаточное количество данных. Помимо модулей, встроенных в систему возможно подключение внешних модулей через специализированный интерфейс. В частности, через него подключаются алгоритмы классификации из библиотеки weka, так как непосредственная интеграция библиотеки, написанной на Java, в C++ программу затруднительна. 

\begin{figure}[h]
\begin{center}
\includegraphics[width=15cm]{img/arch}
\end{center}
\caption{Архитектура системы классификации данных лазерной локации}
\label{arch}
\end{figure}

\subsection{Схема работы}

\subsubsection{Модуль визуальной коррекции и контроля}

Это приложение отвечает за визуализацию облаков точек и значений генерируемых для них признаков. Оно позволяет проводить ручную классификацию данных, необходимую для получения тестовых наборов и, проводить тестирование различных алгоритмов автоматической классификации и визуально определять наиболее проблемные области.

\begin{figure}[!h]
\begin{center}
\includegraphics[width=17cm]{img/screenshots/viewer}
\end{center}
\caption{Интерфейс визуальной коррекции и контроля}
\label{viewer}
\end{figure}
\begin{figure}[!h]
\begin{center}
\includegraphics[width=17cm]{img/screenshots/viewer-test}
\end{center}
\caption{Режим отображения ошибок классификации. Ошибки отображаются яркими точками.}
\label{viewer-test}
\end{figure}

После загрузки данных и выбора набора используемых классов в главном окне (рисунок \ref{viewer}) появляется панель с кнопками, соответствующими классам и отображается облако точек. Пользователь может выделить интересующие его фрагмент и он будет отображен в левой панели. Пользователь может выбирать между одной из двух предустановленных проекций и трехмерной визуализацией. Точки, принадлежащие выделенной области могут быть классифицированны вручную, путем нажатия на кнопку в панели инструментов, или автоматически, путем выбора соответствующего пункта из меню.

В программе предусмотрены функции контроля качества классификации. При вызове автоматического классификатора в режиме "Проверка" вместо переопределения классов точек будет изменена их яркость, в зависимости от того, правильно была классифицирована точка или нет (рисунок \ref{viewer-test}). Кроме того, возможна визуализация (рисунок \ref{features}) значений признаков, доступных в точках, что позволяет оценивать их различимость и подбирать оптимальные параметры вычисления этих  признаков.

\begin{figure}[h]
\centering
\subfloat[]{\includegraphics[width=4cm]{img/screenshots/classes}}
\subfloat[]{\includegraphics[width=4cm]{img/screenshots/intensity}}
\subfloat[]{\includegraphics[width=4cm]{img/screenshots/height}}
\subfloat[]{\includegraphics[width=4cm]{img/screenshots/hvar}}
\caption{Визуализация признаков точек: a - классы, b - яркость, c - высота, d - разброс высот}
\label{features}
\end{figure}

\subsubsection{Подсистема многоэтапной классификации}
Главное окно подсистемы многоэтапной классификации (рисунок \ref{decnet decnet}) позволяет редактировать граф классификации. На левой панели расположены доступные этапы классификации, которые можно перетащить на рабочее пространство. Блоки могут быть соединены с произвольными низлежащими блоками направленными связями. При этом производится автоматическое определение набора признаков, доступного а каждом из этапов. Управление набором признаков (удаление ненужных, переименование, для соответствия входным требованиям блоков) производится с помощью блока фильтрации признаков, диалог настройки которого изображен на рисунке \ref{decnet filter}. Если блок подразумевает классификацию, пользователь может выбрать, на основании каких признаков проводить классификацию и какой признак содержит информацию о классе объекта. В случае использования внешнего модуля классификации, дополнительные параметры передаются посредством аргументов командной строки (рисунок \ref{decnet weka}).

Запуск процесса обучения и классификации производится посредством кнопок "Start" и "Train". По окончании процесса будет напечатана достигнутая точность и дисперсия количества ошибок.

\begin{figure}[h]
\subfloat[]{\includegraphics[width=5cm]{img/screenshots/decnet}\label{decnet decnet}}\qquad
\subfloat[]{\includegraphics[width=5cm]{img/screenshots/weka}\label{decnet weka}}\qquad
\subfloat[]{\includegraphics[width=5cm]{img/screenshots/filter}\label{decnet filter}}
\caption{Интерфейс построения графа многоэтапной классификации}
\label{decnet}
\end{figure}

\subsection{Описание проведенных экспериментов}

Тестирование системы производилось на наборах, описанных в приложении \ref{test data}. Поиск оптимальных параметров алгоритмов производился методом градиентного спуска. Признаковое пространство было составлено из: высоты точки над поверхностью земли, разброса высот в радиусе 2м., яркостью точки, расстояния до аппроксимирующей плоскости, угла наклона аппроксимирующей плоскости. При использовании второго этапа классификации дополнительно использовались метки двух наиболее популярных классов в окрестности, соотношение количества точек этих классов, расстояние до плоскостей, аппроксимирующих точки этих классов и разность между яркостью текущей точки и средней яркостей точек указанных классов. Таким образом учитывались локальные характеристики связных множеств, образуемых прецедентами с одинаковым классом.

Классификация производилась на 5 классов: земля, дорожное покрытие, здание, дерево, кусты. Попытка введения класса автомобилей не увенчалась успехом в силу недостаточной представленности этого класса в тестовых наборах. Результаты тестирования (рисунок \ref{results}) показали, что введение дополнительного этапа, имеющего информацию о расположении объектов, в большинстве случаев позволяет повысить качество классификации, а также значительно сократить количество одиночных ошибок.

\begin{figure}[h]
\centering
\includegraphics[width=17cm]{img/result}
\caption{Результаты измерения качества классификации}
\label{results}
\end{figure}