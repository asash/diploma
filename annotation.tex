\newpage
\section*{Аннотация}

%Аннотация (не более пол-страницы) содержит формулировку задачи и основных результатов

%Данная работа посвящена задаче классификации прецедентов, образующих связные множества, возникающей при классификации результатов измерений в узлах некоторой сетки. Объектам в исходном непрерывном пространстве соответствуют связные области измерений (прецедентов) в дискретном пространстве узлов сетки. Частным случаем этой задачи является задача классификации данных лазерной локации.
%Из-за отсутствия информации о форме и расположении объектов данная задача традиционно решается либо независимой классификацией прецедентов (с введением коллективных признаков), либо с использованием специфичных для свой области применения алгоритмов выделения объектов. В данной работе рассматривается подход, основанный на выделении границ объектов путем предварительной классификации и поиска связных областей прецедентов одного класса.
%В результате работы была разработана экспериментальная система многоэтапной классификации данных лазерной локации. Серия экспериментов, проведенных с помощью этой системы, показала увеличение качества классификации.

Данная задача посвящаена задаче реализации многопрофильной системы фильтрации спама.
Задача традиционно решается при помощи байесовских классификаторов. Существуют два подхода к построению
таких классификаторов - персонифицированный и неперсонифицированный.

В работе представлено решение на базе более сложных методов машинного обучения, которое показывает
лучшие результаты по сравнению с наивным байесовским классификатором и, кроме того, добавлена поддержка
\textbf{многопрофильности}, сочетающей преймущества персонифицированного и неперсонифицированного подходов.

