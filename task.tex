\newpage
\section{Цели работы и постановка задачи}

% Постановка задачи должна содержать формулировку задачи в рамках определенной модели
% предметной области, к которой относится решаемая задача, требования к искомому
% -решению в терминах используемой модели предметной области

Данная работа имеет две основные цели:
\begin{itemize}
\item Реализовать в рамках так из свободных антиспам-систем фильтрацию почты, основанную на методе опорных векторов.
\item Разработать и реализовать в рамках той же системы многопрофильную фильтрацию, основанную на методе опорных векторов.
\end{itemize}

Для этого предлагается решить следующие подзадачи:
\begin{enumerate}
\item  Провести обзор существующих открытых систем фильтрации спама и выбрать средство для расширения. 
\item В рамках выбранного средства реализовать фильтрацию спама, основанную на методе опорных векторов.
\item Разработать на базе метода опорных векторов метод, который позволит классифицировать почтовые сообщения по нескольким профилям.
\item Реализовать данный метод в рамках выбранного ранее средства.
\item Провести экспериментальное исследование на открытых тестовых наборах
\item Произвести апробацию разработанного средства на реальных данных.
\end{enumerate}

Разрабатываемое средство должно удовлетворять следующим требованиям:
\begin{itemize}
\item должно распространяться под свободной лицензией;
\item должно работать демоном на сервере;
\item может быть интегрировано в большинство распространенных MTA;
\item в случае работы для одного пользователя качество классификации не хуже чем у наивного байесовского классификатора;
\item в случае работы для нескольких пользователей в режиме многопрофильной фильтрации качество классификации лучше чем у наивного байесовского классификатора;
\item производительность классификации должна быть сравнима с байесовскими методами.
\end{itemize}
