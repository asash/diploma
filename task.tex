\newpage
\section{Цель работы и постановка задачи}

% Постановка задачи должна содержать формулировку задачи в рамках определенной модели 
% предметной области, к которой относится решаемая задача, требования к искомому 
% -решению в терминах используемой модели предметной области

Целью настоящей работы является изучение особенносей классификации прецедентов, образующих связные множества, и построение системы классификации, использующей эти особенности. В качестве прикладной задачи была выбрана задача классификации данных лазерной локации, так как она является характерным представителем рассматриваемого класса задач.

\subsection{Постановка задачи}
\begin{itemize}
  \item
  Исследовать особенности применения методов машинного обучения к задаче классификации прецедентов, образующих связные множества
  \item
  Предложить способ улучшения качества классификации путем комбинирования методов классификации и более точного учета зависимостей классов связных прецедентов 
  \item
  Реализовать систему классификации данных лазерной локации, использующую предложенный способ
\end{itemize}

\subsection{Требования к решению}

Первоочередным требованием к решению является высокая точность классификации. При этом групповые ошибки, то есть не обнаруженные небольшие объекты, считаются более желательным результатом, чем одиночные ошибки. Это связано с тем, что в случае ручной проверки данных оператору проще создать или удалить один крупный объект, чем большое количество мелких. В случае же использования данных в автоматических системах, хранение и обработка большого количества несуществующих объектов приводит к большим накладным расходам.