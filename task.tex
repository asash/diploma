\newpage
\section{Цель работы и постановка задачи}

% Постановка задачи должна содержать формулировку задачи в рамках определенной модели
% предметной области, к которой относится решаемая задача, требования к искомому
% -решению в терминах используемой модели предметной области
Целью работы является разработка и реализация системы фильтрации спама использующий многопрофильный подход.
Для этого предлагается решить следующие подзадачи:
\begin{enumerate}
\item  Провести обзор существующих открытых систем фильтрации спама и выбрать средство для расширения. Требования к средству:
средство должно распространяться под свободной лицензией;
работает на сервере демоном;
развивается и поддерживается сообществом;
может быть интегрировано в большинство распространенных MTA.
\item Реализовать в рамках данного средства алгоритм классификации (SVM) описанный в работе Розинкина.
\item Разработать модификацию алгоритма SVM, которая позволит классифицировать почтовые сообщения по нескольким профилям.
\item Реализовать данную модификацию в рамках выбранного ранее средства.
\item Провести экспериментальное исследование на тестовых наборах
\begin{itemize}
    \item Spamassassin public corpus
    \item CEAS 2008 Live Spam Challenge Laboratory Corpus
    \item Провести апробацию на реальной почтовой системе (ЛВК).
\end{itemize}
\end{enumerate}


\subsection{Требования к разрабатываемому средству}
\begin{itemize}
\item средство должно распространяться под свободной лицензией
\item средство должно работать демоном на сервере
\item может быть интегрировано в большинство распространенных MTA.
\item в случае работы для одного пользователя качество классификации не хуже чем у наивного байеса
\item в случае работы для нескольких пользователей в режиме многопрофильной фильтрации качество классификации лучше чем у наивного байеса
\item производительность классификации сравнимо с наивным байесом.
\end{itemize}
