\newpage
\section{Обзор существующих решений}
\label{review}
% Обзор должен содержать явно сформулированные цели и  критерии сравнения, 
% которые должны коррелировать с требованиями к искомому решению исходной задачи.
% В конце обзора должны быть сформулированы выводы.

Одна из проблем, возникающих при решении рассматриваемой задачи, заключается в том, что признаки каждого отдельного прецедента описывают признаки областей, к которым они относятся, лишь опосредованно. Причем, чем меньше шаг сетки измерения, тем меньшую область исходного объекта описывает прецедент и тем вероятнее, что значения признаков в нем будут значительно отклоняться от характерных для данного объекта. Кроме того, существует целый ряд признаков, которые могут быть определены только для объекта в целом, в частности это форма и размеры объекта, или только в некоторой окрестности прецедента, например конечные разности признаков (с.м. приложение\ref{local properties} ), измеренных в самих прецедентах. Стоит отметить, что такие признаки могут иметь высокую описательную способность, например небольшие строения очень часто имеют характерную форму крыши с несколькими скатами. Как правило скаты располагаются под углом близким к $45^\circ$ к горизонтали, редко встречающимся у объектов другого типа\cite{full waveform}.

Но при вычислении таких признаков возникает проблема выбора множества прецедентов, по которым производится вычисление признаков. Обычно в качестве такого множества выбирается сфера или куб для трехмерных данных или пространство ограниченное цилиндрической поверхностью с круглым или квадратным основанием для псевдотрехмерных\footnote{прецеденты располагаются над некоторой плоскостью, причем над одной точкой плоскости может находиться только один прецедент. Хорошим примером псевдотрехмерных данных являются данные авиационной лазерной локации} данных. Оптимальный радиус определяется либо опытным путем, либо динамически подстраивается под фрагменты сцены\cite{SMAP}.

Если радиус слишком мал, получаемые признаки будут иметь слишком большую дисперсию, что приводит к ухудшению результатов. С другой стороны, при таком выборе множеств, на которых вычисляются признаки, на границе объектов они вычисляются по прецедентам, принадлежащим нескольким объектам, что лучшем случае приводит к отнесению прецедентов не к тому объекту, а в худшем к выявлению нового не существующего объекта. Так перепад высот на границах зданий часто классифицируется как деревья\cite{full waveform}. Кроме того, при таком выборе областей невозможно вычислять признаки, описывающие объект в целом. Для этого требуется требуется использование алгоритмов выделения объектов. 

Для данных авиационной лазерной локации в качестве таких методов используются модификации алгоритма роста регионов\cite{Brenner, growing}, или комбинация алгоритма выделения точек, принадлежащих поверхности земли\cite{filer comparison}, и методов выделения связных областей. Состоятельность такого подхода для простых сцен была показана в \cite{whole objects}. Недостатком этих алгоритмов является их ориентированность на конкретную задачу. В обоих случаях выделение объектов производится на основе формы поверхности, в первую очередь перепадов высот. Это не позволяет находить плоские объекты, такие как дороги (автомобильные и железные), пляжи, луга, болота, выходы скальных пород. Возможна модификация, учитывающая другие характеристики прецедентов, но и она не будет универсальной.

Для данных наземной лазерной локации, в силу большей разрешающей способности и возможности получения данных с нескольких точек, возможно использование алгоритмов восстановления связности между точками путем поиска близлежащих точек. В качестве объектов выбираются подмножества точек с большой степенью связности\cite{MMM}. Этот метод имеет те же самые недостатки, ограничивая набор классов объектами, хорошо различимыми по геометрической форме.

\subsection{Учет зависимостей между прецедентами}

Вторая проблема классификации прецедентов, образующих связные множества, заключается в наличии зависимостей между прецедентами. Эти зависимости могут иметь две причины: 
\begin{itemize}
 \item Прецеденты, относящиеся к одному связному множеству имеют сходные характеристики т.к. описывают один объект, в том числе эта связь может выражаться в признаках, не вносящих большого вклада в точность классификации. Например, автомобиль может быть покрашен в любой цвет, поэтому не может быть отличен от других объектов только по цвету. Однако, точки принадлежащие одному автомобилю будут почти всегда иметь один и тот же цвет.
 \item Зависимость между прецедентами может быть обусловлена наличием зависимостей между объектами. Так, на данных авиационной лазерной локации средней точности автомобили плохо отличимы от зарослей кустарника. Но автомобили почти всегда располагаются на дорогах. 
\end{itemize}
В большинстве случаев для классификации выбираются наиболее простые алгоритмы, не учитывающие зависимостей между прецедентами. Тем не менее, они показывают относительно неплохие результаты благодаря использованию коллективных признаков. Эти признаки неявным образом учитывают зависимости между прецедентами за счет усреднения значений в некоторой окрестности.

Альтернативный подход, нашедший применение в последнее время заключается в использовании графических моделей, в частности попарных Марковских сетей. Они позволяют учитывать зависимости между прецедентами в явном виде:
$$
P(l)=\frac{1}{Z}\prod_{i=1}^{N}\phi_i(l_i)\prod_{d(i,j)<\varepsilon}\phi_{i,j}(l_i,l_j)
$$
\begin{ESKDexplanation}
\item[где ] $P(l)$ "--- вероятность того, что прецеденты будут иметь метки l;
\item $N$ "--- количество прецедентов;
\item $\phi_i(l_i)$ "--- потенциал прецедента $i$ принимать метку $l_i$;
\item $\phi_{i,j}(l_i,l_j)$ "--- потенциал прецедентов $i$ и $j$ одновременно принимать метки $l_i$ и $l_j$ соответственно;
\item $d(i,j)$ "--- расстояние между точками $i$ и $j$.
\end{ESKDexplanation}

Соответственно, эти алгоритмы требуют дополнительного шага вычисления зависимостей. Ставится задача построения регрессии для совместного распределения вероятности для двух прецедентов. Как правило она решается с помощью стандартных методов машинного обучения, однако для задачи классификации данных лазерной локации зачастую искусственно завышается вероятность того, что соседние точки будут иметь одинаковый класс\cite{Markov random fields}. Этот прием направлен использование априорного знания о том, что прецеденты формируют связные множества с одним классом, а также на подавление единичных ошибок\footnote{То есть прецедентов, имеющих класс отличный от подавляющего большинства близлежащих прецедентов} которые значительно труднее исправлять вручную, чем групповые ошибки.

Дальнейшим развитием этого направления может являться использование графических моделей для совместная классификация прецедентов и объектов. Не смотря на то, что такие методики применяются в других областях\cite{RDN}, упоминаний их применения к классификации данных лазерной локации обнаружено не было.

\begin{figure}[h]
\begin{center}
\includegraphics{img/RDN}
\end{center}
\caption{Структура зависимостей при совместной классификации прецедентов и объектов}
\label{RDN}
\end{figure}

\subsection{Методы оценки качества классификации}
Наиболее распространенной оценкой точности классификации является
$$
P=\frac{N_t}{N}
$$
\begin{ESKDexplanation}
\item[где ] $P$ "--- точность классификации;
\item $N_t$ "--- количество верно классифицированных прецедентов;
\item $N$ "--- общее число прецедентов.
\end{ESKDexplanation}
Она присутствует во всех работах, связанных с данной тематикой, что позволяет производить сравнения с ними. Но эта оценка не учитывает количества классов и имеет тенденцию отдавать предпочтение более часто встречающимся классам. Стоит отметить, что цена ошибки между различными классами, вообще говоря, может быть разной. Так, для данных лазерной локации стандартом ASPRS LAS предусмотрены классы низкой и высокой растительности, не имеющие четкой границы. Поэтому, для более детального анализа точности, дополнительно строится матрица ошибок, вычисляется точность распознавания каждого класса и средняя точность распознавания класса.

Второй критерий качества классификации заключается в степени сложности получаемых данных. Одиночные ошибки на этапе выделения объектов превращаются в объекты крошечного размера. Они могут мешать интерпретировать данные, так как интерпретация может полагаться на априорные знания о размерах и значимости объектов. Примером такого несоответствия может служить здание размером $1\mbox{м}^2$ на автодороге. Кроме того, большое количество ошибочных объектов создает большие накладные расходы на их обработку.

Удаление таких ошибок с помощью стандартных фильтров импульсного шума, таких как медианный фильтр или метод голосования, приводит к потере детализации. Поэтому желательно использование методов при удалении единичных ошибок учитывающих помимо класса прецедентов зависимости между ними и значения признаков.

Для измерения качества классификации в смысле количества одиночных ошибок сцена делится на блоки фиксированного размера, для каждого блока вычисляется точность классификации и вычисляется дисперсия точности классификации блоков. Чем выше дисперсия, тем лучше. В идеальных условиях эта оценка не зависит от точности классификации, что позволяет использовать ее для сравнения алгоритмов, имеющих различную точность. Практически, тестовый набор ограничен и, чем точнее алгоритм, тем меньше количество ошибок, а значит растет статистическая погрешность оценки количества одиночных ошибок.

$$
D=\sum_B(P_b-P)^2
$$
\begin{ESKDexplanation}
\item[где ] $D$ "--- качество классификации в смысле количества одиночных ошибок;
\item $B$ "--- множество блоков;
\item $P_b$ "--- точность классификации в блоке $b \in B$;
\item $P$ "--- точность классификации.
\end{ESKDexplanation}

\begin{figure}[h]
\centering
\subfloat[]{\includegraphics[width=7cm]{img/stage1}}\qquad
\subfloat[]{\includegraphics[width=7cm]{img/stage2}}
\caption{Ошибки классификации. a: точность 83\%, дисперсия 158. b: точность 84\%, дисперсия 168}
\end{figure}

\subsubsection{Выбор обучающих и тестовых наборов. Скользящий контроль}
\label{datasets abscense}
Несмотря на хорошую доступность данных авиационной лазерной локации \cite{national dataset}, не существует общепринятых тестовых наборов для оценки качества классификации. Отчасти это вызвано быстрым совершенствованием техники и желанием использовать данные более высокого качества для улучшения результатов классификации. В связи с этим различные авторы используют не просто различные наборы данных, а наборы различающиеся своей детализацией и набором признаков. Соответственно, прямое сравнение точности алгоритмов, предлагаемых авторами не возможно.

Второй особенностью оценки точности классификации прецедентов, образующих связные множества, является зависимость оценки точности от способа разделения данных на тестовый и обучающие наборы, в том числе и при использовании скользящего контроля. Из-за наличия зависимостей между прецедентами при \textbf{случайном} разделении на тестовый и обучающий наборы, они также оказываются зависимыми и имеют схожие характеристики. В результате различие между оценками точности на тестовом и тренировочном наборе оказывается в пределах статистической погрешности. Более корректным является выбор в качестве тестового и тренировочного наборов данных, полученных из не пересекающихся областей исходного пространства. К сожалению, авторы статей зачастую не сообщают, как именно они проводили тестирование.

Однако, даже такой подход не гарантирует независимость наборов. Она может заключаться например в наличии в обоих наборах коттеджных поселков с типовыми коттеджами. Эта проблема может быть решена только введением общепринятых тестовых наборов. Чтобы как-то её смягчить, в данной работе приводятся результаты тестирования на трех наборах: обучающем, схожим с обучающим тестовом и тестовым, отличающимся от обучающего. Кроме того, помимо предлагаемого подхода к классификации, были протестированы многие классические алгоритмы, что обеспечивает корректность сравнения, так как все измерения проводились в одинаковых условиях.

\subsection{Выводы}

Исследование показало, что существующие на данный момент методы классификации прецедентов, образующих связные множества, не в полной мере используют эту особенность, и не учитывают, что эти множества соответствуют объектам в реальном мире, которые тоже имеют признаки. Это приводит к большому количеству ошибок классификации на краях объектов и большому количеству одиночных ошибок. Такое положение дел объясняется отсутствием сведений о связности прецедентов, а также расположении и форме объектов, в исходных данных.