\newpage
\section{Обзор предметной области}
\label{review}

\subsection{Методы фильтрации спама}
\subsubsection{схема работы E-Mail}
Для того, чтобы понять где и как можно бороться со спамом необходимо разобраться с общей схемой работы электронной почты, которая определена в \cite{RFC2081}. Пусть пользователь $A$ отправляет пользователю $B$ письмо. Ниже приведена типичная последовательность действий, которые выполняются между моментами времени, в который $A$ нажимает кнопку ``отправить'' в своем почтовом клиенте и моментом времени, когда $B$ видит письмо в своем почтовом клиенте. Пусть для определенности $A$ имеет адрес foo@bar.net, а $B$ имеет почтовый адрес b@example.com.

\begin{figure}[h]
\begin{center}
\includegraphics[width=10cm]{img/emailscheme}
\end{center}
\caption{Схема работы электронной почты. Меджду двумя агентами доставки почты (MTA) может находиться еще некоторое количество промежуточных серверов.}
\label{email_scheme}
\end{figure}

\begin{enumerate}
\item почтовый клиент(Mail User Agent, MUA) пользователя $A$ инициирует соединение с сервером электронной почты (Mail Transporting Ageint, MTA) домена bar.net и передает ему письмо для b@example.com по протоколу SMTP.
\item Почтовый сервер домена bar.net запрашивает у своего DNS сервера информацию о домене example.com и выясняет адрес сервера, ответственного за обработку писем для домена example.com
\item Почтовый сервер домена bar.net инициирует соединение с почтовым сервером домена example.com и передает ему письмо по протоколу SMTP
\item Почтовый сервер домена example.com анализирует письмо, выясняет что домен адресата соответствует домену, который обслуживает этот сервер и иницирует соединение с агентом доставки почты (Mail Delivery Agent, MDA). Способ передачи письма от MTA к MDA не определен стандартом(например может использвоваться протокол доставки локальной почты LMTP - Local Mail Transfer Protocol).
\item MDA передает письмо в хранилище. Письмо хранится в хранилище до тех пор, пока почтовый клиент пользователя $B$ не инициирует соединение с ним и не заберет у него письмо. Для взаимодействия хранилища почты с почтовым клиентом обычно используются такие протоколы как IMAP4 или POP3
\end{enumerate}

\subsubsection{Где можно бороться со спамом}
Исходя из схемы работы электронной почты бороться со спамом можно в следующих местах:
\begin{itemize}
\item Компьютер отправителя. В последнее время большая часть спама рассылается с компьютеров, зараженным вредоносным программным обеспечением. Дл того чтобы этого не происходило на компьютере отправителя можно устанавливать различные антивирусы, блокировщики  траффика и другое ПО, помогающее защититься от вредоносных программ.
\item Шлюз провайдера. Провайдер отправителя может блокировать подключения, которые ,по его мнению, устанавливаются с целью рассылки спама.
\item Отправляющий SMTP-сервер. Может отказаться от передачи сообщения, если сочте, что письмо является спамом.
\item Принимающий SMTP сервер. Может отказаться от приема сообщения, может уведомить получателя о попытке передачи спама. Может лучше фильтровать сообщения с учетом особенностей конкретных пользователей.
\item Принимающий компьютер. Может проанализировать письмо и пометить его как спам. Необходима настройка фильтрации для каждой установки почтовой программы.

Фильтрация спама на потчовом сервере получателя представляет наибольший интерес, так как она имеет все достоинства фильтрации в почтовой программе получателя, и кроме того избавляет пользователя от необходимости \textbf{получать} нежелательные письма. В дальнейшем будем рассматривать именно фильтрацию на почтовом сервере получателя.
\end{itemize}

\subsubsection{Как можно бороться со спамом}
Можно выделить несколько больших групп методов борьбы со спамом. Эти группы различаются по виду информации о входящем сообщении, которую они используют в своей работе. Рассмотрим некоторые из них, в порядке увеличения знаний о сообщении

\textbf{Сетевой уровень}

Существуюи методы, которые позволяют фильтровать часть спама еще на этапе попытки установления соединения с сервером. Достоинством этих методов является низкое использование системных ресурсов, однако ни один из них не способен отфильтровать 100 \% спама.
\begin{itemize}
\item \textbf{Блокировка спамерских IP-адресов}. Раньше спамеров было не много, и они рассылали почту с одних и тех же машин. Эффективным методом борьбы было блокирование соединений с IP-адресов, которые администратор системы посчитал спамерскими.

\item \textbf{DNSBL}\cite{MCMILLAN}. Этот подход как и предыдущий базируется на блокировке сообщений со спамерских IP-адресов. Отличие состоит в том, что вместо локальной базы данных спамерских IP-адресов используется удаленная база данных, за достоверность которой отвечает обслуживающая её организация. Проверка происходит следующим образом: при попытке подключения сервер преобразует IP-адрес инициатора подключанеия в специальное доменное имя по определенным в стандарте правилам (это имя должно указывать на некий поддомен домена принадлежащего организации). После этого сервер пытается разрешить это имя при помощи стандартного протокола DNS/ Если при разрешении доменного имени определился адрес 127.0.0.1, это означает что IP-адрес инициатора соединения в удаленной базе данных нет. Если же определился какой-либо другой адрес, то адрес в удаленной базе есть, соединение нужно заблокировать.

\item \textbf{Обратный DNS + проверка имени}. Обычно почтовые серверы серьезных организаций имеют характерные доменные имена (например mx4.google.com или mail.lvk.cs.msu.su). В то же время компьютеры обычных пользователей либо вообще не имеют имен, либо имеют доменные имена, по которым их можно опознать, как автоматически выданные провайдером. (например v10-159-159.yandex.net или 156-234-mytischi.net) Для определения доменного имени инициатора подключения можно выполнить обратный dns-запрос. Дальше можно произвести анализ полученного имени (например при помощи регулярных выражений) и отсеять имена, принадлежащие обычным пользователям (т. к. легитимная почта должна передаваться через почтовый сервер). 

\item \textbf{Sender Policy Framework}. \cite{SPF} SPF - это расширение протокола SMTP\cite{RFC2081}, позволяющее в специальной записи для домена указать список серверов, имеющих право отправлять письма с обратным адресом в этом домене. SPF позволяет проверить, не был ли подделан обратный адрес и заблокировать прием письма в противном случае.
\end{itemize}

\textbf{Протокол SMTP}

Стандартный протокол используемый для передачи электронной почты - SMTP (Simple Mail Transfer Protocol). Описание этого протокола можно найти в стандартах RFC \cite{RFC2081}. Анализируя общение клиента и сервера по протоколу SMTP можно отфильтровывать часть спама.


\begin{figure}[h]
\begin{center}
\includegraphics[width=10cm]{img/smtp_session}
\end{center}
\caption{Пример сессии SMTP}
\label{smtp_session}
\end{figure}

\begin{itemize}
	\item \textbf{Нарушения протокола}. В RFC\cite{RFC2081} определена последовательность команд, при помощи которых общаются клиент и сервер (см. рисунок\ref{smtp_session}). Легитимные почтовые агенты обычно четко следуют этому протоколу, в то время как ПО для рассылки спама часто от него отклоняется. Например такое ПО может передавать серверу команду за командой, не дожидаясь ответа от сервера. Таким образом, сервер получивший следующую команду не отправив ответ на предыдущую  может определить клиента как вредоносное ПО и отказаться от приема сообщения.
	\item \textbf{GreyListing} \cite{GREYLISTING}. По стандарту SMTP \cite{RFC2081} сервер может сообщить о возникновении временной ошибки. Это означает, что в данный момент сервер не может принять письмо, но в будущем ситуация может измениться и письмо будет принято. Сервер, который использует технологию серых списков, первоначально отклоняет любое письмо от неизвестного отправителя, сообщаяя о временной ошибке. В базу данных записывается информация об этой попыткуе. Легитимный сервер положит такое письмо в очерендь и позже повторит поптыку отправки письма. При этом его адрес будет перемещен в белый список и все последующие письма с этого сервера будут проходить без задержек. ПО для рассылки спама не имеет такой функциональности, поэтому второй попытки передачи письма не будет.
	
Недостатком этого метода являются задержки, которые возникают при попадании писем от легимных пользователей в серый список.
\end{itemize}

\subsection{Обзор методов машинного обучения} 
\subsubsection{Обучение по прецедентам}
Пусть есть неизвестная функция $F: X -> Y$, переводящая объекты
множества $X$ в объекты множества $Y$, причем для некторых $x_1, x_2, ... x_n$ известны соответсвующие им значения $y_1 = F(x_1), y_2 = F(x_2), ... y_n = F(x_n)$.
Необходимо построить функцию $F^*(X)$, наилучшим образом приближающую $F(X)$.

Под фразой \textbf{наилучшим образом приближает} подразумевается, что для некоторого функционала качества $\mu(y, y')$, матожидание
\begin{equation}
\label{}
E\mu(F(x), F^*(x)), x \in X
\end{equation}
будет минимальным.

В качестве $\mu(y, y')$ можно брать например модуль разности, квадрат разности и т. п.

\textbf{Методом обучения} называется проецесс построения $F^*(x)$ по известным парам $(x_1, y_1), (x_2, y_2), ... (x_n, y_n)$. Множество таких пар называется \textbf{обучающей разности}

Построенную функцию $F^*(x)$ часто также называют \textbf{алгоритмом}, подразумевая что она должна быть эффективно вычислима на компьютере.

Задача построения такого алгоритма не может быть решена точно, т. к. неизвестна природа исходной функции $F(x)$. Кроме того минимизация $\mu$ на обучающей выборке не обязательно приводит к тому, что и на новых обхектах из $X$ значение $\mu$ также будет мало.
%TODO смотри пример(рисунок, переобучение)

Процесс построения модели можно также рассматривать как выбор конкретного значения функции $F^*(x)$ из семейства $F^*(x, \pi)$. Значение выбранного параметра $\pi$ в таком случае называется \textbf{моделью} для алгоритма
$F^*(x)$

В целом процесс решения задачи машинного обучения называют \textbf{машинным обучением} или \textbf{обучением по прецедентам}

В зависимости от вида множества $Y$, задача может являться задачей классификации(множество конечно) или регрессии(множество бесконечно).
Задача фильтрации спама по своей сути является задачей классификации с двумя классами.

Объекты из множества $X$ обычно рассматривают как вектор в некотором $N$-мерном пространстве. Если изначально объекты имеют более сложную структуру(например текст, аудиозапись), то ее некоторым способом представляют в виде вектора. Элементы таких векторов называются \textbf{признаками}.

\subsubsection{Скользящий контроль}
Так как вид функции $F(X)$  неизвестен, то прямо посчитать значение матожидания \ref{matozh} не возможным. Для того чтобы хоть как-то оценить качество построенного алгоритма обычно пользуются методом \textbf{скользащего контроля}. Метод заключается в следующем: первоначальная обучающая выборка делится на несколько частей. Обучение проводится по очереди на каждой из частей, а значение $\mu$ оценивается по оставшимся частям.

Итоговую оценку $\mu$ считают как
\begin{equation}
\mu' = 1/n\sum_1^n{\mu_i}, i \in 1..n
\end{equation}

\subsubsection{Байесовские методы}
\subsubsection{Наивный байесовский классификатор}
\subsubsection{Нейронные сети}
Нейронная сеть — это математическая модель, а также ее программные или аппаратные реализации, построенная в некотором смысле по образу и подобию сетей нервных клеток живого организма.

Нейронные сети — один из наиболее известных и старых методов машинного обучения.

\subsubsection{Решающие деревья}
Решающие деревья представляют собой идейно достаточно простой метод обучения: строится дерево, в узлах которого ставятся некоторые предикаты(простые пороговые решающие правила). В листах этого дерева записываются значения $F^*(x)$, соответсвующие значеиниям предикатов.

Обычно такие деревья склонны к переобучению, однако существуют закрытые реализации алгоритмов классификации, в которых деревья используются в качестве базовых алгоритмов для более сложных алгоримов.
\begin{figure}[h]
\begin{center}
\includegraphics[width=10cm]{img/d_tree}
\end{center}
\caption{Решающее дерево: в узлах предикаты, в листьях значения целевой функции}
\label{d_tree}
\end{figure}

\subsubsection{Метод опорных векторов}
Метод опорных векторов основан на \textbf{принципе максимизации зазора}. Пусть стоит задача классификации с двумя классами. В пространстве признаков классифицируемых объектов проводится гиперплоскость таким образом, чтобы объекы обучающей выборки принадлежащие одному классу лежали по одну сторону от этой гиперплоскости, а объекты принадлежащие второму классу по другую.

Понятно, что если гиперплоскость существует, то она не единственна. Среди всех таких гиперплоскостей выбирается такая, которая максимизирует \textbf{зазор} -- минимальное расстояние от гиперплоскости до ближайшей точки обучающей выборки. Таким образом расстояние от гиперплоскости до ближайшего объекта каждого из классов будет одинаково.

\begin{figure}[h]
\begin{center}
\includegraphics[width=5cm]{img/svm}
\end{center}
\caption{Наилучшая разделяющая прямая в двухмерном пространстве}
\label{svm}
\end{figure}

Если разделяющая гиперплоскость существует выборка называется \textbf{линейно разделимой}.

Для обобщения на случай линейно неразделимой выборки используется следующая идея: выборку можно сделать линейно разделимой, если увеличить размерность пространства. Для этого используется так называемая \textbf{ядерная функция}

\begin{figure}[h]
\begin{center}
\includegraphics[width=15cm]{img/svm2}
\end{center}
\caption{Неразделимая в одномерном пространстве выборка стала разделимой после перевода в двумерное пространство}
\label{svm-kernel}
\end{figure}

Метод опорных векторов в настоящее время рассматривается как наиболее универсальный и хорошо работающий на большом количестве задач. Кроме того в работе А. Розинкина было показано что этот метод показывает хорошие результаты при применении его к задаче фильтрации спама. Для решения поставленной задачи мы также воспользуемся этим методом.

\subsection{Обзор существующих открытых систем фильтрации спама}
По требованию постановки задачи реализация должна быть выполнена в виде модификации одной из существующих систем фильтрации спама. В данном разделе будут приведены описания некоторых систем фильтрации спама и выбрано одно из них для дальнейшей модификации.
\subsubsection{spamassassin}
SpamAssassin - одно из наиболее известных открытых средств фильтрации спама. Этот проект динамично развивается и показывает хорошие результаты производительности и качества фильтрации. SpamAssassin использует в своей работе большое количество методик обнаружения спама.
\subsubsection{spambrobe}


\subsubsection{dspam}
\textbf{dspam} - открытая система фильтрации спама. Dspam изначально проектриовался для работы в многопользовательском режиме.
Для фильтрации спама dspam может использовать одну из нескольких разновидностей байесового классификатора.
Dspam написан на языке С и работает достаточно эффективно. Dspam имеет большое сообщество разработчиков и активно развивается в настоящий момент.
Из недостатков системы dspam можно отметить использование устаревшей системы сборки (autools) и использование низкоуровневой разработки, что усложняет понимание и модификацию его исходного кода.
Для доработки был выбран именно dspam, так как он удовлетворяет нашим требованиям, а именно:
\begin{itemize}
\item Ориентирован на работе на стороне сервера
\item Распространяется под свободной лицензией
\item Ориентирован на многопользовательский режим
\item Фильтрация спама осуществляется  всего одним алгоритмом, что упрощает тестирование разработанного метода
\end{itemize}
