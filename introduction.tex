\newpage
\section{Введение}

%Введение должно описывать предметную область, к которой относится задача, решаемая в дипломной работе, 
% содержать неформальное ее описание;

Задача классификации прецедентов, образующих связные множества, возникает когда требуется выделить и классифицировать объекты в непрерывном реальном пространстве, однако, в процессе извлечения данных происходит переход к набору величин, измеренных в узлах некоторой сетки (рисунок \ref{points}). Объекту в непрерывном пространстве соответствует связная область измерений (прецедентов) в дискретном. Соответственно, в отличии от классической задачи классификации класс является свойством не отдельного прецедента, а множества прецедентов. 

\begin{figure}[!b]
\begin{center}
\includegraphics{img/points}
\end{center}
\caption{Точки, образующие объекты}
\label{points}
\end{figure}

Главной проблемой, возникающей при решении этой задачи является то, что в процессе измерения информация о связности и границах областей зачастую теряется. В результате их непосредственная классификация становится невозможной. Однако, так как прецеденты являются измерениями, расположенными в узлах некой сетки, известно расстояние между ними.

\subsection{Способы решения задачи классификации прецедентов, образующих связные множества}
\subsubsection{Введение коллективных признаков}
Вводятся признаки, вычисляемые на основе всех прецедентов, расположенных внутри шара заданного радиуса с центром в рассматриваемом прецеденте. Такие признаки описывают локальные свойства пространства. Если радиус этого шара значительно меньше типичного размера связной области, признак в большинстве случаев вычисляется на прецедентах одной области и описывает её локальные свойства. Такие признаки обладают хорошей описательной способностью\cite{Amin}, однако приводят к ошибкам на границах связных областей, так как в таком случае они вычисляются по прецедентам сразу нескольких областей.

Чем больше выбирается размер области, по которой вычисляется признак, тем лучшей описательной способностью он обладает, но тем больше доля приграничных прецедентов, на которых он не работает.
\subsubsection{Оценка корреляции соседних прецедентов}
Второй подход заключается в оценке для каждой пары прецедентов вероятности того, что они принадлежат одному связному множеству. Эта оценка в дальнейшем применяется при классификации с помощью таких методов, как Марковские сети \cite{Markov random fields, MMM}.

\subsection{Лазерная локация}
\begin{wrapfigure}{r}{9 cm}
	\includegraphics[width=9cm]{img/lidar.png}
	\caption{Схема работы авиационного лазерного локатора}
\end{wrapfigure}
Лазерная локация использует лазерный дальномер для определения расстояния до точки, на которую направлен луч и вращающееся зеркало для сканирования окружающего пространства. Таким образом определяются координаты точек, принадлежащих объектам этого пространства. При сопряжении и системами глобального и инерциального позиционирования становится возможным получать координаты точек в системе координат, связанной с Землей. Наибольший интерес представляет авиационная лазерная локация, так как она позволяет за короткое время получать информацию о больших площадях.
Однако для дальнейшего использования полученных данных зачастую требуется их классификация. \textbf{Эта задача является частным случаем задачи классификации прецедентов, образующих связные множества}. В роли прецедентов выступают точки ландшафта, на которые попал лазерный луч. Известны характеристики этих точек (см. приложение \ref{point properties}). Требуется определить к какому типу объектов принадлежит каждая точка. Таким образом класс является свойством не точки (прецедента), а связного объекта (связного множества точек).

\subsection{Актуальность}

Лазерная локация является одним из передовых и наиболее динамично развивающихся способов сбора геопространственной информации. С её помощью составляются подробные карты местности, карты высот, определяются характеристики растительного покрова, расположение и форма природных и антропогенных объектов. Одними из её ключевых преимуществ являются оперативность и высокая точность. Плотность измерений может достигать 70 $\mbox{точек}/\mbox{м}^2$. Для выделения геопространственной информации из этих данных требуется их обработка на СВТ, заключающаяся в устранении шумов, классификации (как правило на землю, дороги, растительность, здания, воду) и ряда других шагов, зависящих от стоящих целей. На сегодняшний день классификация, как правило, производится посредством применения узкоспециализированных алгоритмов, определение параметров запуска и порядок применения которых полностью лежит на операторе АРМ, что приводит к относительно невысокой эффективности (5 $\mbox{км}^2/\mbox{день}$). Данная работа посвящена разработке системы автоматической классификации данных лазерной локации на основе методов машинного обучения, что должно значительно улучшить оперативность обработки данных.

Сама классификация как правило не является конечной целью, однако является неотъемлемой частью решения многих прикладных задач, опирающихся на лазерную локацию.

\subsubsection{Составление карт местности}
В следствии природных катаклизмов или активной деятельности человека, карты могут потерять свою актуальность и требуется их обновление. План местности есть ни что иное, как отклассифицированная поверхность Земли с пометками о характеристиках объектов отдельных классов (лиственный лес или хвойный), информацией о высотах и географических названиях. Все это, за исключением географических названий, может быть извлечено из данных лазерной локации в автоматическом (как правило со средним качеством) или автоматизированном (с хорошим качеством) режиме. То есть актуальный и подробный план местности может быть получен непосредственно после пролета самолета.

\subsubsection{Проверка состояния ЛЭП}
Потребность современного мира в электроэнергии возрастает. Для обеспечения подачи все большего количества энергии приходится либо строить новые линии электропередач, либо менять режимы функционирования существующих. При этом меняются требования по допустимым расстояниям до объектов различного типа. Одним из наиболее простых способов проверки выполненности новых требований  является съемка линий электропередач с помощью лазерных локаторов с последующей классификацией полученных точек и выявления мест, в которых расстояние от проводов до точек заданных классов меньше допустимого. 

\subsubsection{Имитационное моделирование}

Для задач имитационного моделирования требуется создание модели окружающей среды. Одним из способов получения такой модели является лазерная локация. В частности она успешно применяется для моделирования и предсказания наводнений\cite{flood}. При этом возникает необходимость в выделении и классификации объектов, расположенных на местности с целью определения их сопротивления потоку воды.