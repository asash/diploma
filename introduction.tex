\newpage
\section{Введение}

%Введение должно описывать предметную область, к которой относится задача, решаемая в дипломной работе,
% содержать неформальное ее описание;


\subsection{Задача фильтрации спама}
Данная работа посвящена фильтрации спама. Под спамом обычно понимают массовую рассылку сообщений. Такими рассылками могут быть рекламные материалы приходящие по обычной почте, SMS,  через системы обмена мгновенными сообщениями, по электронной почте.

Электронная почта в настоящее время является одним из основных способов общения в сети Интернет. В протокол доставки электронной почты \cite{RFC2081} \cite{RFC2822} при его разработке не были включены никакие средства проверки подлинности личности отправителя, что существенно облегчило рассылку  сообщений. Кроме того, поскольку копирование электронного сообщения практически бесплатно, при использовании электронной почты проблема спама стоит особо остро. В данной работе под термином \textbf{спам} будет подразумеваться нежелательная электронная почта, то есть такая почта, которую пользователь не хотел бы получать даже зная о факте её отправки.

Термином \textbf{легитимная почта} мы будем обозначать электронные письма, не являющиеся спамом.

Так как получение нежелательных сообщений  отвлекает пользователей и создает ненужную нагрузку на сеть, со спамом необходимо бороться. С момента появления спама было придумано множество методов позволяющих отличить спам от легитимной почты. Перечислим некоторые классы этих методов:
\begin{enumerate}
\item Методы, использующие информацию о сетевом соединении. В своей работе они анализируют IP-адреса, доменные имена и другую информацию на сетевом уровне.
\item Методы работающие на уровне протокола передачи электронной почты. Эти методы используют особенности протокола передачи почтовых сообщений (SMTP)  и точность его соблюдения клиентом.
\item Анализ заголовков и тела письма. Эти методы используют информацию содержащуюся в самом письме: комбинации заголовков, ключевые слова, соответствие регулярным выражениям. Могут применяться различные статистические методы.
\end{enumerate}
Подробное описание методов было приведено в курсовой работе \cite{PETROV10}. 

В настоящее время системы, рассылающие спам научились достаточно хорошо имитировать работу корректных почтовых сервисов и таким образом обходить фильтры первых двух типов. Поэтому, практически все современные антиспам-системы так или иначе используют статистические фильтры, так как они позволяют наиболее качественно отсеивать нежелательные сообщения. В данной работе остановимся именно на статистических методах.
\subsection{Статистические методы}

Статистические методы можно представить в виде «черного ящика», получающего на вход письмо, и на выходе возвращающего оценку принадлежности этого письма к спаму. Внутри черного ящика выполняется алгоритм, который выдает оценку для письма в зависимости от некоторых статистических данных. Наборов данных может быть несколько, например отдельный набор данных для каждого адресата. Совокупность таких данных для одного адресата будем называть \textbf{профилем}.

Профиль строится  при помощи \textbf{обучающей выборки} – некоторого количества сообщений, для которых известно являются они спамом или легитимной почтой. \textbf{Обучением} статистического метода называется процесс построения профиля по обучающей выборке. Поскольку со временем содержание спам-рассылок имеет тенденцию меняться, необходимо поддерживать профиль пользователя в актуальном состоянии, то есть перестраивать его с учетом новых данных. Для некоторых алгоритмов необходимо полностью повторять процесс обучения, однако существуют алгоритмы, в которых можно использовать информацию полученную на ранних этапах добавив к ним новые знания. Процесс добавления в профиль новой информации называется \textbf{дообучением}.

Существует два подхода к построению профиля пользователя, необходимого для работы статистических методов:

\begin{itemize}
\item В обучающей выборке содержатся письма от всех пользователей. В этом случае при добавлении нового пользователя в систему он сразу может начинать пользоваться фильтром, не помечая приходящие к нему письма как спам или  легитимная почта. Однако возникает другая проблема, связанная с тем, что схожие письма для одних пользователей являются спамом, а для других легитимной почтой. (Например, менеджер вполне может получать письма с коммерческими предложениями, при этом если такие письма получает программист, то они почти наверняка являются спамом.)
\item Все письма из обучающей выборки принадлежат одному пользователю. В этом случае алгоритм работает достаточно хорошо (отсеивая практически весь спам, при этом вероятность ложного срабатывания достаточно мала). Однако, для того, чтобы алгоритм начал работать пользователь должен получить и добавить в свою обучающую выборку достаточно большое количество спама и легитимной почтой.
\end{itemize}
Оба метода обладают недостатками, и, по этому, является актуальной задача разработки подхода, комбинирующего достоинства обоих подходов.


\subsection{Размещение системы фильтрации  спама}
Фильтрацию спама можно проводить либо на стороне клиента, либо на стороне сервера. 
При фильтрации на стороне клиента почтовый клиент скачивает письмо с сервера, после этого классифицирует почту. 

Основным достоинством метода является то, что пользователь сам может решить нужно ли ему фильтровать спам, как его фильтровать.

К недостаткам метода можно отнести необходимость настройки фильтрации на каждой машине пользователя, создание  дополнительной нагрузки на компьютерную сеть, невозможность использовать неперсонифицированный подход к фильтрации спама.


При фильтрации на стороне сервера письмо классифицируется до передачи его в почтовый клиент пользователя. При этом пользователь может даже не скачивать письмо. Достоинствами метода являются настройка спам-фильтра только в одном месте(на почтовом сервере), низкая нагрузка на сеть, возможность использовать как персонифицированный, так и неперсонифицированный подход к настройке спама.

В дальнейшем будет рассматриваться только серверная фильтрация спама, так как такой подход обладает рядом достоинств по сравнению с клиентской фильтрацией, а недостатки не являются существенными. Кроме того, как будет показано в разделе \ref{research}, разрабатываемый в данной работе метод можно применить только на почтовом сервере.



\subsection{Методы машинного обучения}
\label{MLINTRO}
Фильтрация спама статистическими методами тесно связана с теорией машинного обучения. Многие термины приведенные выше при описании статистического подхода к фильтрации спама были заимствованны из этой теории.


\textbf{Машинное обучение} – это научная дисциплина, изучающая построение алгоритмов, способных обучаться\cite{VORONCOV}. При помощи машинного обучения решаются такие задачи как восстановление регрессии, кластеризация, классификация, прогнозирование.  

Задача фильтрации спама является \textbf{задачей классификации}: нужно определить принадлежность письма к одному из классов: спам или легитимная почта.

Формально задача классификации ставится следующим образом: имеется множество \textbf{объектов}, разделённых некоторым образом на классы. Для конечного подмножества объектов известно, к каким классам они относятся. Классовая принадлежность остальных объектов не известна. Требуется построить алгоритм, способный классифицировать произвольный объект из исходного множества. 

\textbf{Классифицировать} объект – значит указать номер(или наименование) класса, к которому принадлежит объект. Классификация объекта – номер или  наименование класса, выдаваемое алгоритмом классификации в результате его применения к данному конкретному объекту.

\textbf{Обучением} называется процесс постройки алгоритма классификации по обучающей выборке. \textbf{Дообучение} – процесс перестроения алгоритма классификации после добавления в обучающую выборку новых объектов. Для многих алгоритмов их не приходится перестраивать их с ноля, можно воспользоваться информацией полученной при ранних этапах обучения. Такие алгоритмы называются \textbf{дообучаемыми}.

Некоторые из методов машинного обучения подверженны проблеме \textbf{переобучения}. Проблема заключается в том что алгоритм может пытаться находить закономерности в обучающей выборке там где их на самом деле нету.

Задачу классификации можно поставить \textbf{вероятностно}. В этом случае построенный  алгоритм должен выдавать не конкретный класс, а вероятность принадлежности объекта тому или иному классу.

Существует достаточно большое множество методов машинного обучения(см. например \cite{VORONCOV}) 

\subsection{Актуальность разработки многопрофильного средства фильтрации спама}

Все существующие на сегодняшний день открытые средства фильтрации спама(имеются ввиду развивающиеся и поддерживаемые системы) используют либо персонифицированный, либо неперсонифицированный подход к фильтрации. Персонифицированный подход при должном обучении может обеспечить более качественную фильтрацию, однако процесс обучения необходимо повторять для каждого нового пользователя почтовой системы. Поэтому, как было показано выше, является актуальной разработка третьего подхода, сочетающего достоинства обоих подходов. 

Такой подход в рамках данной работы мы будем называть \textbf{многопрофильным} 

В курсовой работе \cite{PETROV10} был реализован многопрофильный подход на основе байесовских классификатороров(о фильтрации спама байесовскими методами см. \cite{PLANFORSPAM} и \cite{ROBINSON} ). В работе А. Розинкина\cite{ROZ} было показано, что применив более сложные методы машинного обучения(в частности метод опорных вектров) можно получить более качественный классификатор. В данной работе приведен метод, основанный на методе опорных векторов поддерживающий многопрофильность.
