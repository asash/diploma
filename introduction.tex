\newpage
\section{Введение}

%Введение должно описывать предметную область, к которой относится задача, решаемая в дипломной работе,
% содержать неформальное ее описание;


\subsection{Задача фильтрации спама}
Данная работа посвящена фильтрации спама. Под спамом обычно понимают массовую рассылку сообщений, как то рекламные материалы приходящие по обычной почте, SMS,  системы обмена мгновенными сообщениями,  электронной почте.

Электронная почта в настоящее время является одним из основных способов общения. В протокол доставки электронной почты при его разработке не были включены никакие средства проверки подлинности личности отправителя, что существенно облегчило рассылку  сообщений. Кроме того, поскольку копирование электронного сообщения практически бесплатно, при использовании электронной почты проблема спама стоит особо остро. В данной работе под термином спам будет подразумеваться нежелательная электронная почта, то есть такая почта, которую пользователь не хотел бы получать даже зная о факте её отправки.

Термином легитимная почта мы будем обозначать электронные письма, не являющиеся спамом.

Так как получение нежелательной почты отвлекает пользователей электронной почты и создает ненужную нагрузку на сеть, со спамом необходимо бороться. С момента появления спама было придумано множество методов позволяющих отличить спам от легитимной почты. Перечислим некоторые классы этих методов:
\begin{itemize}
\item Методы, использующие информацию о сетевом соединении. В своей работе они анализируют IP-адреса, доменные имена и другую информацию на сетевом уровне.
\item Методы работающие на уровне протокола передачи электронной почты. Эти методы используют особенности протокола передачи почтовых сообщений (SMTP)  и точность его соблюдения клиентом.
\item Анализ заголовков и тела письма. Эти методы используют информацию содержащуюся в самом письме: комбинации заголовков, ключевые слова, соответствие регулярным выражениям. Могут применяться различные статистические методы.
\end{itemize}

\subsection{Статистические методы}

Эти методы представляют  из себя «черный ящик», получающий на вход письмо, и на выходе возвращающий оценку принадлежности этого письма к спаму. Внутри черного ящика выполняется алгоритм, выдающий оценку для письма в зависимости от некоторых статистических данных. Совокупность таких данных для некоторого адресата будем называть профилем.

Профиль строится  при помощи обучающей выборки – некоторого количества сообщений, для которых известно являются они спамом или легитимной почтой. Обучением статистического метода называется процесс построения профиля по обучающей выборке. Поскольку постоянно появляются новые виды спама и легитимной почты, необходимо уметь перестраивать профиль пользователя с учетом новых данных. Для некоторых алгоритмов необходимо полностью повторять процесс обучения, однако существуют алгоритмы, в которых можно использовать информацию полученную на ранних этапах добавив к ним новые знания. Процесс добавления в профиль новой информации называется дообучением.

Существует два подхода к построению профиля пользователя , необходимого для работы статистических методов:

\begin{itemize}
\item В обучающей выборке содержатся письма от всех пользователей. В этом случае при добавлении нового пользователя в систему он сразу может начинать пользоваться фильтром, не помечая приходящие к нему письма как спам или  легитимная почта. Однако возникает другая проблема, связанная с тем, что схожие письма для одних пользователей являются спамом, а для других легитимной почтой. (Например, менеджер вполне может получать письма с коммерческими предложениями, при этом если такие письма получает программист, то они почти наверняка являются спамом.)
\item Все письма из обучающей выборки принадлежат одному пользователю. В этом случае алгоритм работает достаточно хорошо (отсеивая практически весь спам, при этом вероятность ложного срабатывания достаточно мала). Однако, для того, чтобы алгоритм начал работать пользователь должен получить и добавить в свою обучающую выборку достаточно большое количество спама или легитимной почтой.
\end{itemize}

Все существующие на сегодняшний день открытые средства фильтрации спама используют либо первый, либо второй подход к фильтрации. Данная работа посвящена разработке третьего подхода, комбинирующего достоинства этих методов. В этом подходе должны строиться профили для каждого пользователя  и некоторые общие профили для групп пользователей. При классификации письма мы будем пользоваться некоторой комбинацией профилей. Такой подход мы будем называть многопрофильным.

\subsection{Размещение системы фильтрации  спама}
Фильтрацию спама можно проводить либо на стороне клиента, либо на стороне сервера.  При фильтрации на стороне клиента почтовый клиент скачивает письмо с сервера, после этого классифицирует почту.
При фильтрации на стороне сервера письмо классифицируется до передачи его в почтовый клиент пользователя. При этом пользователь может даже не скачивать письмо. Многопрофильную фильтрацию можно провести можно произвести только на стороне сервера, т. к. только на стороне сервера хранится информация,  необходимая для построения множества профилей пользователей.

\subsection{Методы машинного обучения}
Алгоритм, реализующий статистический метод по своей сути реализует тот или иной метод машинного обучения.
Машинное обучение – это научная дисциплина, изучающая построение алгоритмов, способных обучаться. При помощи машинного обучения решаются такие задачи как восстановление регрессии, кластеризация, классификация, прогнозирование.

Задача фильтрации спама является задачей классификации, нужно определить принадлежность письма к одному из классов: спам или легитимная почта.

Формально задача классификации ставится следующим образом: Имеется множество объектов (ситуаций), разделённых некоторым образом на классы. Задано конечное множество объектов, для которых известно, к каким классам они относятся. Это множество называется обучающей выборкой. Классовая принадлежность остальных объектов не известна. Требуется построить алгоритм, способный классифицировать произвольный объект из исходного множества. Классифицировать объект – значит указать номер(или наименование) класса, к которому принадлежит объект. Классификация объекта – номер или  наименование класса, выдаваемое алгоритмом классификации в результате его применения к данному конкретному объекту.

Обучением называется процесс постройки алгоритма классификации по обучающей выборке. Дообучение – процесс перестроения алгоритма классификации после добавления в обучающую выборку новых объектов. Для многих алгоритмов их не приходится перестраивать их с ноля, можно воспользоваться информацией полученной при ранних этапах обучения. Такие алгоритмы называются дообучаемыми.

Задачу классификации можно поставить вероятностно. В этом случае вместо класса объекта алгоной раритм должен выдавать не конкретный класс, а вероятность принадлежности объекта тому или иному классу.

Существует большое множество методов машинного обучения, обзор которых будет приведен в данной работе.
