\newpage
\section{Заключение и результаты}

% Заключение (не более чем на 1 страницу) должно содержать краткую формулировку
% результатов работы, выносимых на защиту и согласованных с целью работы.

В рамках данной	работы были решены следующие задачи:
\begin{itemize}
\item Произведен обзор существующих свободных систем фильтрации спама.
\item Реализована фильтрация сапама на основе метода опорных векторов в рамках открытого средства.
\item На основе метода опорных векторов разработан подход к фильтрации спама, позволяющий классифицировать сообщения по нескольким профилям.
\item В рамках выбранного открытого средства фильтрации спама реализован такой подход.
\item Произведено экспериментальное исследование, подтвердившее работоспособность метода.
\item Произведена апробация метода на реальных данных.
\end{itemize}


Произведенное экспериментальное исследование на реальном почтовом трафике показало эффективность разработанного средства, поэтому в настоящее время производится подготовка к его промышленному внедрению в почтовую систему ЛВК.

Разработанное реализовано в виде модификации системы dspam. Осмысленным является включение этих модификаций в основное дерево разработки проекта. Для этого необходимо совместно с разработчиками dspam оформить изменений в соответствии с требованиями предъявляемыми к исходному коду проекта dspam. 
