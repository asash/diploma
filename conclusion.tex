\newpage
\section{Заключение и результаты}

% Заключение (не более чем на 1 страницу) должно содержать краткую формулировку
% результатов работы, выносимых на защиту и согласованных с целью работы.

В рамках данной	работы были решены следующие задачи:
\begin{itemize}
\item Произведен обзор существующих открытых систем фильтрации спама.
\item Реализована фильтрация сапама на основе метода опорных векторов в рамках открытого средства.
\item На основе метода опорных векторов разработан подход к фильтрации спама, позволяющий классифицировать сообщения по нескольким профилям.
\item В рамках выбранного открытого средства фильтрации спама реализован такой подход.
\item Произведено экспереминтальное исследование, подтвердившее работоспособность метода.
\end{itemize}

\subsection{Перспективы развития}
Разработанное средство позволяет классифицировать письма по многим профилям при помощи метода опорных векторов. Теория машинного не стоит на месте и постоянно появляются новые методы, которые позволяют решать задачу классификации лучше чем раньше. Вероятно, применяя более сложные методы можно получить более совешенный спам-фильтр.

Разработанное средство реализовано на основе системы dspam. Для этого было произведено ответвление от основн ого дерева исходных кодов и произведена модификация подсистемы фильтрации спама. В перспективе возможно вливание произведенной модификации в основную ветку проекта dspam. 
