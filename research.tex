\newpage
\section{Исследование и построение решения задачи}
\label{research}
% В разделе «Исследование и построение решения задачи» должна быть проведена
% декомпозиция исходной задачи на последовательность подзадач, которые нужно
% решить для получения решения исходной задачи, приведены обосновании всех
% принимаемых решений. Например, если принимается решение о создании некоторого
% программного средства, то необходимо показать, что не существует средства,
% обладающего нужными характеристиками. Исключение составляет случай, когда
% такое средство создается в учебных целях. Обоснование может быть дано одним
% из следующих способов:
% 1.Экспертный: приводятся высказывания, мнения авторитетных специалистов, с
% указанием ссылок на источники, где оно сформулировано;
% 2.Дедуктивный: яркий пример математика - есть система аксиом и правил вывода.
% Если ты сумел показать, как вывести свое утверждение из аксиом с помощью
% правил вывода, то все обосновано.
% 3.Естественнонаучный: выдвигается
% гипотеза (то, что обосновываем) и проводится серия экспериментов, на основании
% обработки результатов этих экспериментов гипотеза либо подтверждается, либо нет
% 4. Инженерно-практический:  хорош когда в качестве утверждения выступает некий
% принцип или система, работоспособность которого мы хотим обосновать, тогда
% экспериментальная реализация может выступать в качестве обоснования..

Для того, чтобы применить метод опорных векторов к задаче фильтрации спама,
необходимо научиться представлять письма в виде, пригодном для применения
этого метода - в виде вектора признаков.
Кроме того, нам необходимо построить многопрофильную систему, а для этого
классифицируемый вектор должен содержать какую-то информацию о пользователе.

\subsection{Выделение лексем}
Под \textbf{лексемами} мы будем понимать последовательности символов, на которые мы разбиваем исходный текст. В некотором смысле лексема - это аналог слова. Выделить слова из текста зачастую бывает затруднительно, кроме того зачастую смысловую нагрузку несет лишь часть слова, а иногда напротив лишь последовательность слов. Для разбиения текста на лексемы существует несколько способов:
\begin{itemize}
\item \textbf{Разбиение до пробельных символов или знаков препинания}. В этом случае лексема получается практически полным аналогом слова. Например текст ``Как хорошо, замечательно жить в этом мире!'' разобьется на следующие лексемы:
 ``Как'' , ``хорошо'', ``замечательно'', ``жить'', ``в'', ``этом'' и ``мире''.

 \item \textbf{Выделение n-грамм}. В этом случае из текста выделяются все цепочки содержащие ровно n символов. Например при n=6 фраза ``hello, world'' породит следующие 6-граммы: ``hello,'', ``ello, '', ``llo, w'', ``lo, wo'', ``o, wor'', ``, worl'' и `` world''.

 \item \textbf{Выделение цепочек}. Использование в качестве лексем цепочек из нескольких слов. При этом в качестве лексем выделяются как сами слова, так и последовательности из нескольких подряд идущих слов. Например ``free viagra'' породит лексемы ``free'', ``viagra'' и ``free viagra''.
\item {Выделение частичных цепочек}. Этот алгоритм рассматривает окна из нескольких слов, и генерирует цепочки состоящие из некоторых (необязательно идущих подряд) слов этого окна. Например, из  текста ``the quick brown fox jumped'' при размере окна 4 будут сгенерированы такие, как ``the quick brown *'', ``quick * * jumped'', ``* * * jumped'' и другие.

\item {Выделение биграмм}. Алгоритм похож на использование частичных цепочек, однако генерирует только те цепочки, в которых содержится ровно два слова. Для предыдущего примера алгоритм сгенерирует цепочки ``the quick * *'' ``* * brown jumped'', ``the * * fox''  и другие.
\end{itemize}

В dspam можно по желанию использовать любой из алгоритмов, но по умолчанию используется метод выделения цепочек. Разработчики dspam указывают\cite{TOKS} что применительно к задаче фильтрации спама такой метод работает не хуже других. Для решения нашей задачи мы также воспользуемся этим методом.

Так как термин \textbf{лексема} по смыслу близок к слову в обычном понимании, то в дальнейшем говоря о частотах слов мы будем подразумевать частоты лексем.

\subsection{Представление письма в векторном виде}
\label{MESSAGEVECTOR}
Метод опорных векторов работает с объектами, представленными в векторном виде.
Письма имеют текстовую, структуру. Однако, у электронного сообщения возможно выделить большое количество
числовых признаков, например:
\begin{itemize}
    \item Количество слов в письме;
    \item Дата отправления;
    \item Количество заголовков;
    \item Количество вложений.
\end{itemize}
Все эти признаки в принципе могут быть использованы для построения векторного
представления письма, однако они никак не передают самое важное, что отличает
спам от легитимной почты - смысл сообщения.

\begin{figure}[h]
\begin{center}
\includegraphics[width=10cm]{img/vectorize}
\end{center}
\caption{Векторное представление текста}
\label{svm-kernel}
\end{figure}

Человек, для того, чтобы понять смысл текста, анализирует совокупность слов в этом тексте.  Хорошей характеристикой тематики текста являются частоты входящих в него слов.
Таким образом мы можем взять все слова в языке, посчитать количество каждого из них в письме
и получить вектор частот.

Проблема заключается в том, что размерность такого вектора чрезмерно высока, причем
почти все признаки в этом случае окажутся нулевыми. Русский язык содержит по некоторым оценкам более 500000\cite{TUTIN} слов, а с
учетом словоформ количество слов может превысить несколько миллион.

Скорость работы метода опорных векторов линейно зависит размерности пространства классифицируемых векторов.
Большая размерность вектора приведет к увеличению времени построения модели и времени классификации.
Кроме того метод склонен к переобучению при больших размерностях векторов. Поэтому необходимо каким-то образом ограничить количество используемых для классификации слов.

Для решения задачи выбора используемых для классификации была произведена серия экспериментов.
Были протестированы следующие способы отбора:
\begin{itemize}
\item Выбор наиболее частых слов
\item Выбор случайно отобранных слов
\item Выбор слов для которых разница между частотой появления в спаме и в легитимной почте максимальна
\end{itemize}

Экспериментальным путем было установлено что наилучшие результаты метод опорных векторов проявляет при использовании 1500 самых частых слов. При меньшем количестве информации слишком мало для классификации, при большем - начинают проявляться проблемы переобучения.

\subsection{Многопрофильность}
\label{MULTIPROFILE}
Нам необходимо построить такой классификатор, который с одной стороны для классификации пользуется данными от  всех пользователей, а с другой стороны учитывает специфику адресата письма.  Это обозначает, что используемый метод обучения должен использовать некоторую информацию об адресате письма. 

В качестве информации, которую можно использовать для классификации удобно использовать некоторый числовой идентификатор, который уникален для каждого пользователя почтовой системы. Если добавить такой идентификатор к векторному представлению письма, то одинаковые письма для разных пользователей будут различаться. 

Рассмотрим ситуацию, при которой определенный класс писем (векторные представления которых достаточно близки) один пользователь считает спамом, а другой легитимной почтой.  В этом случае добавив в признаковое описание письма идентификатор адресата мы дадим классификатору информативный признак, позволяющий отличить спам от легитимной почты. Если метод обучения должным образом учтет этот признак при построении модели, то в последствии построенный классификатор будет различать приходящие письма по этому признаку и классифицировать их правильно. 

Теперь рассмотрим другую ситуацию, при которой определенный класс писем все пользователи почтовой системы считают спамом (или легитимной почтой). Пусть в обучающей выборке есть некоторое количество писем из этого класса от разных пользователей. В этом случае признаки, соответствующие лексемам, входящим в такие (и только такие) письма будут иметь высокую информативность, и классификатор при построении модели сможет учесть это обстоятельство.

Пусть письмо из этого класса пришло пользователю, который еще не добавлял в свою выборку писем из этого класса. В этом случае, так как признаки, соответствующие лексемам из этого письма имеют большую информативность, такое письмо также правильно определится как спам (или соответственно как легитимная почта).

Таким образом добавление к векторному представлению письма числового идентификатора позволит решить задачу построения многопрофильного спам-фильтра, если метод обучения корректно обработает такой признак. 

Уникальный идентификатор адресата в почтовой системе является \textbf{номинальным} признаком (т. е. признаком, который может принимать лишь конечное количество различных значений). К сожалению, метод опорных векторов (как и многие другие методы, применяемые для решения задачи классификации) не всегда корректно обрабатывает номинальные признаки.\cite{YURYSVM} Для решения этой проблемы применяется \textbf{бинаризация}.

Пусть есть номинальный признак $a$, способный принимать одно из $n$ значений $a_1, a_2, ... a_n$. При бинаризации такой признак заменяется на $n$ признаков $b_1, b_2, ... b_n$, причем $b_i$ равно $1$, если исходный признак равен $a_i$ и $0$ в противном случае.

\begin{figure}[h]
\begin{center}
\includegraphics[width=7cm]{img/add_uid}
\end{center}
\caption{Добавление информации о профиле в вектор письма. Письмо адресовано второму пользователю.}
\label{multiprofile}
\end{figure}


После бинаризации признака, соответствующего идентификатору адресату метод опорных векторов сможет корректно обрабатывать такой признак и успешно справляться с классификацией сообщений в многопрофильном режиме.

Таким образом, общая схема построения признакового описания письма такова:

\begin{itemize}
	\item Строится вектор частот.
	\item Строится вектор, идентифицирующий пользователя.
	\item Два вектора конкатенируются, получается результирующий вектор.
\end{itemize}


