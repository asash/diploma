\newpage
\section{Исследование и построение решения задачи}
\label{research}
% В разделе «Исследование и построение решения задачи» должна быть проведена 
% декомпозиция исходной задачи на последовательность подзадач, которые нужно 
% решить для получения решения исходной задачи, приведены обосновании всех 
% принимаемых решений. Например, если принимается решение о создании некоторого 
% программного средства, то необходимо показать, что не существует средства, 
% обладающего нужными характеристиками. Исключение составляет случай, когда 
% такое средство создается в учебных целях. Обоснование может быть дано одним 
% из следующих способов:
% 1.Экспертный: приводятся высказывания, мнения авторитетных специалистов, с 
% указанием ссылок на источники, где оно сформулировано;
% 2.Дедуктивный: яркий пример математика - есть система аксиом и правил вывода. 
% Если ты сумел показать, как вывести свое утверждение из аксиом с помощью 
% правил вывода, то все обосновано.
% 3.Естественнонаучный: выдвигается 
% гипотеза (то, что обосновываем) и проводится серия экспериментов, на основании 
% обработки результатов этих экспериментов гипотеза либо подтверждается, либо нет
% 4. Инженерно-практический:  хорош когда в качестве утверждения выступает некий 
% принцип или система, работоспособность которого мы хотим обосновать, тогда 
% экспериментальная реализация может выступать в качестве обоснования.. 

Задача классификации данных лазерной локации сводится к задаче классификации прецедентов, образующих связные множества, путем выполнения подготовительных этапов: загрузки данных, обеспечения быстрого доступа к ним, определения используемых признаков и их вычисления. Эти подготовительные этапы являются специфичными для задачи. В отличии от них, этапы классификации и выделения объектов универсальны, поэтому сформулированы в терминах прецедентов и связных областей.

\subsection{Загрузка данных}

На сегодняшний день существует два распространенных формата хранения данных лазерной локации: LAS и текстовый. Чаще всего используется текстовый формат, представляющий из себя CSV\footnote{Значения перечислены через точку с запятой, записи разделены переносами строк} в котором заданы координаты и яркости точек\footnote{Точное число полей и их содержание зависит от модели и производителя оборудования}. Главными недостатками этого формата являются крайне неэффективное хранение данных и непостоянная длинна записи, что исключает возможность чтения файла с произвольного места. Альтернативой является стандарт LAS, хранящий данные в бинарном виде и имеющий фиксированную длину записи. Кроме того стандарт LAS определяет набор классов, к которым может принадлежать точка, впрочем предусмотрена возможность использования собственного набора классов. В виду простоты конвертации текстовых данных в LAS и наличия библиотеки для работы с ним именно этот формат был выбран в качестве входного.

Данные лазерной локации могут содержать миллионы точек, а для вычисления коллективных признаков необходимо быстро находить все точки, расположенные в некотором ограничивающем объеме. В литературе предлагается два решения: регуляризация данных и KD-деревья. Регуляризация заключается в рассмотрении данных как псевдотрехмерных, после чего точки выравниваются по равномерной двумерной сетке\cite{Amin}. Недостающие значения интерполируются и данные заносятся в двумерный массив. Таким образом обеспечивается обращение по координатам и поиск соседей за константное время. Однако этот метод не применим для трехмерных данных, и приводит к большим накладным расходам или неточностям в случае неравномерности распределения точек. К сожалению даже данные авиационной лазерной локации не всегда удовлетворяют этим условиям. KD-деревья более универсальны и могут быть использованы для любых данных. В итоге было выбрано комбинированное решение: данные делятся на блоки, хранящиеся в многомерном массиве, а внутри блока точки хранятся в KD-дереве. Это позволяет снизить высоту деревьев на несколько уровней, но главное преимущество такого решения заключается в возможности использования блоков для поточной классификации.

Для каждого блока определяется, какие соседние блоки необходимы для вычисления коллективных признаков точек блока. Когда поступают все необходимые соседние блоки, блок ставится в очередь на классификацию. Также возможна повторная классификация блока, если поступила дополнительная информация для него или его соседей. Кроме того, добавление нового блока не приводит к балансировке деревьев в других блоках, такая локальность изменений позволяет параллельно получать и обрабатывать данные.

\subsection{Выделение объектов}

Как было сказано в обзоре, в настоящий момент объекты, как правило, выделяют исходя из геометрических соображений.
Предлагается вместо специализированной процедуры выделения объектов использовать методы классификации с последующим выделением связных областей, имеющих один и тот же класс. Такой подход позволяет выделять соприкасающиеся объекты разных классов и объекты, не выделяющиеся своей формой, более того, он может применяться и в случае, если не определено расстояние между прецедентами или это расстояние не является мерой. Также возможно представление объектов как нечетких множеств прецедентов. Единственным условием применимости данного подхода является наличие алгоритмов классификации прецедентов.

После выделения объектов становится возможно введение дополнительных признаков, явное использование объектов в алгоритмах классификации и уточнение областей, по которым вычисляются коллективные признаки. Если коллективный признак обладает свойством аддитивности, то он может вычисляться по формуле
$$
F_c=\sum_{p \in N}\sum_{i \in C}\sum_{j \in C}(F^p \cdot p_i \cdot p_j^p)
$$
\begin{ESKDexplanation}
\item[где ] $F_c$ "--- скорректированное значение признака;
\item $N$ "--- множество прецедентов, принадлежащих окрестности прецедента, для которого вычисляется признак;
\item $C$ "--- множество допустимых классов;
\item $F^p$ "--- вклад в признак прецедента $p$;
\item $p_i$ "--- вероятность того, что прецедент, для которого вычисляется признак, имеет класс $i$;
\item $p_j^p$ "--- вероятность того, что прецедент $p$ имеет класс $j$.
\end{ESKDexplanation}

\subsection{Классификация}

Для реализации вышеописанного алгоритма требуется проведение как минимум двух этапов классификации. Но для повышения точности количество этапов может быть увеличено: на различных этапах могут применяться разные наборы классов, производиться отсев заведомо ошибочных объектов, вычисление и уточнение значений коллективных признаков. Поэтому вместо применения фиксированной последовательности алгоритмов предлагается ввести понятие \textbf{многоэтапной классификации}. 

Из операций классификации, оценки зависимостей и иных действий над данными строится ациклический ориентированный граф. Этапом называется выполнение одного из действий, результатом которого является вычисление новых свойств прецедентов, уточнение уже имеющихся свойств или добавление/удаление прецедентов. На каждом этапе доступны все данные, полученные на предыдущем этапе, что позволяет вычислять и использовать взаимосвязи между прецедентами. Состав классов и тип прецедентов может меняться по ходу вычисления.
%TODO вставить картинку с этапами

Соответственно возникает потребность в создании создание программного средства, обеспечивающего такой процесс классификации. Оно должно обеспечивать построение графа классификации пользователем, хранение промежуточных данных (прецедентов) в предметно-независимом виде, передачу данных для обработки алгоритмам, обеспечивающим выполнение этапов, и обучение алгоритмов классификации, реализующих отдельные этапы.

\subsubsection{Обучение многоэтапного классификатора}

Благодаря отсутствию циклов в графе классификации, в случае использования жадной стратегии\footnote{То есть на каждом этапе алгоритмы классификации обучаются на получение наилучшей точности на своем этапе.}, он может быть обучен за один проход сверху-вниз. Для обучения необходимо иметь обучающую выборку, состоящую из прецедентов и назначенных им классов, для каждого их этапов. Обучающая выборка состоит из двух компонент: набора прецедентов и назначенных им классов. В данном случае эти компоненты разделяются: назначенные классы пользователь должен задать для каждого этапа отдельно\footnote{Для всех этапов, на которых список допустимых классов и тип прецедентов совпадают, назначенные классы могут быть заданы одинаковыми.}, а набор прецедентов задается только для первого этапа. Этих данных достаточно для обучения первого этапа. После обучения каждого очередного этапа производится классификация прецедентов обучающей выборки и полученные данные используются для обучения следующего этапа.

\subsubsection{Классификация данных лазерной локации}

Для проведения многоэтапной классификации данных лазерной локации предлагается ввести операции, перечисленные на рисунке \ref{stages}. Специфичными для данной задачи являются модули генерации признаков и специализированные алгоритмы фильтрации, удаляющие одиночные ошибки. Модули генерации признаков можно разделить на две категории: модули для выделения коллективных признаков из множеств прецедентов (список признаков такого типа приведен в приложении \ref
{local properties}) и модули генерации признаков, обеспечивающих передачу информации между этапами.

Использование признаков, описывающих классы близлежащих прецедентов, позволяет учитывать зависимости между ними в методах классификации независимых прецедентов, скорость которых выше скорости работы алгоритмов, учитывающих зависимости в явном виде.

\begin{figure}[t]
\begin{center}
\includegraphics{img/stages}
\end{center}
\caption{Типы операций, используемые при многоэтапной классификации данных лазерной локации}
\label{stages}
\end{figure}

%Декомпозиция задачи. Входные данные - LAS + конвертеры, обосновать почему LAS. Вычисление первичных признаков, определение и вычисление сводных признаков. Классификация. Возможность классификации потока информации. Идея многоэтапной классификации.

%TODO создание обучающих наборов, если будет не хватать объема