\newpage
\section{Исследование и построение решения задачи}
\label{research}
% В разделе «Исследование и построение решения задачи» должна быть проведена
% декомпозиция исходной задачи на последовательность подзадач, которые нужно
% решить для получения решения исходной задачи, приведены обосновании всех
% принимаемых решений. Например, если принимается решение о создании некоторого
% программного средства, то необходимо показать, что не существует средства,
% обладающего нужными характеристиками. Исключение составляет случай, когда
% такое средство создается в учебных целях. Обоснование может быть дано одним
% из следующих способов:
% 1.Экспертный: приводятся высказывания, мнения авторитетных специалистов, с
% указанием ссылок на источники, где оно сформулировано;
% 2.Дедуктивный: яркий пример математика - есть система аксиом и правил вывода.
% Если ты сумел показать, как вывести свое утверждение из аксиом с помощью
% правил вывода, то все обосновано.
% 3.Естественнонаучный: выдвигается
% гипотеза (то, что обосновываем) и проводится серия экспериментов, на основании
% обработки результатов этих экспериментов гипотеза либо подтверждается, либо нет
% 4. Инженерно-практический:  хорош когда в качестве утверждения выступает некий
% принцип или система, работоспособность которого мы хотим обосновать, тогда
% экспериментальная реализация может выступать в качестве обоснования..

